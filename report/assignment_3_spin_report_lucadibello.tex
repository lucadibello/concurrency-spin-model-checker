\documentclass[a4paper, 11pt]{article}
\usepackage[top=3cm, bottom=3cm, left = 2cm, right = 2cm]{geometry} 
\geometry{a4paper} 
\usepackage[utf8]{inputenc}
\usepackage{textcomp}
\usepackage{graphicx} 
\usepackage{amsmath,amssymb}  
\usepackage{bm}  
\usepackage[pdftex,bookmarks,colorlinks,breaklinks]{hyperref}  
\hypersetup{linkcolor=black,citecolor=black,filecolor=black,urlcolor=black} % black links, for printed output
\usepackage{memhfixc} 
\usepackage{pdfsync}  
\usepackage{fancyhdr}
\usepackage{listings}
\pagestyle{fancy}
\setlength{\headheight}{13.6pt}

\title{Model checking with SPIN \\[1ex] \large Software Analysis, Spring 2024}
\author{Luca Di Bello}
\date{\today}

\begin{document}
\maketitle
\tableofcontents

\section{Introduction}

In this assignments examines the implementation of model checking using the \href{https://spinroot.com/spin/whatispin.html}{SPIN} tool to verify the correctness of two versions of a frequency counter program: one sequential and the other parallel. Model checking is a technique that allows to verify the correctness certain properties of a system described in a finite-state model.
\vspace{1em}
\noindent In the following sections will be discussed how the program has been modeled using \href{https://en.wikipedia.org/wiki/Promela}{ProMeLa} language, which Linear Temporal Logic (LTL) properties have been defined to verify the correctness of the program, and how the verification has been performed using SPIN.

\pagebreak

\section{ProMeLa Model}

The ProMeLa model consists of two main processes: the first process handles the sequential computation of frequency counts, storing results in an array \texttt{sequential\_counts}. The second process on the other hand, initiates parallel computation, spawning worker processes for each possible value in the input array. These workers update an array \texttt{parallel\_counts} concurrently. Race conditions are avoided as each worker updates a unique position in the array.

As explicitly stated in the assignment, the ProMeLa model presents two constrants:

\begin{enumerate}
	\item \texttt{MAX}: It represents the maximum value that can be assigned to an element in the array. Used while filling the input array with random values.
	\item \texttt{LENGTH}: the length of the input array.
\end{enumerate}

The model presents an \texttt{init} block that initializes the input array with random values between 0 and \texttt{MAX} and starts both the sequential and parallel processes. The code is available in listing \ref{lst:promela_init}.

\begin{lstlisting}[language=Promela, caption={ProMeLa array intiialization and start of sequential and parallel processes}, captionpos=b, breaklines=true, label={lst:promela_init}]
// Define the maximum number of elements in the array
#define MAX 2
#define LENGTH 2

// Define the variables
int a[LENGTH];

// Keep track of result of both versions of the program
int sequential_counts[MAX + 1];
int parallel_counts[MAX + 1];

// Entry point of the program
init {
	// Initialize the array non-deterministically
  printf("Random state:\n")
	int i;
	for (i : 0 .. LENGTH - 1) {
		// Select a random value for the array
		int v;
		select(v : 0 .. MAX);
		// Assign the value to the array
		a[i] = v;

    // Print the value
    printf("\ta[%d] = %d\n", i, v);
	}

	// Run the sequential version of the program
  printf("Running sequential version...\n");
	run sequentialCounter();
 
  // Run the parallel version of the program
  printf("Running parallel version...\n");
  run parallelCounter();
}
\end{lstlisting}

\noindent As is possible to see from the code above, both versions of the program have been started using the \texttt{run} command inside the same ProMela model. This is different from the Java implementation, where the two versions were implemented in separate classes. This design choice was essential to allow the verification of both versions of the program in the same model as otherwise, it would be impossible to verify by confronting data yielded by two different simulations.

In the abstraction, the sequential and parallel processes are implemented as separate processes. An in depth analysis of each process is provided in the following sections.

\subsection{Sequential Process}

The sequential version of the frequency counter program is implemented in the \texttt{sequentialCounter} process. This process iterates over the input array and increments the corresponding position in the \texttt{sequential\_counts} array.

The ProMeLa implementation of this version of the program is almost identical to the Java implementation. This was expected, as both languages uses a C-like syntax, and the logic of the program is simple. The code is available in listing \ref{lst:promela_sequential}.

\begin{lstlisting}[language=Promela, caption={ProMeLa sequential version of the frequency counter program}, captionpos=b, breaklines=true, label={lst:promela_sequential}]
proctype sequentialCounter() {
	int maxFrequency = -1;
	int k;
	for (k : 0 .. LENGTH - 1) {
		int value = a[k];
		sequential_counts[value] = sequential_counts[value] + 1;
		if
			:: sequential_counts[value] > maxFrequency -> 
				maxFrequency = sequential_counts[value];
				sequential_result = value;
      :: else -> skip;
		fi
	}
  // Signal that the sequential version is done
  sequentialDone = 1;
}
\end{lstlisting}

The main difference between the Java and ProMeLa implementation is the way the program yields the result: in the Java implementation, the result is simply returned by the method, while in the ProMeLa implementation, the result is stored in a global variable \texttt{sequential\_result}. This is necessary as the process cannot return a value.

In addition, I needed to add an additional global variable \texttt{sequentialDone} needed to verify the completion of the sequential process via LTL properties. Learn more about this in section \ref{sec:ltl_properties}.

\subsection{Parallel Process}

The parallel version of the frequency counter program is implemented in the \texttt{parallelCounter} process. This process spawns \texttt{MAX + 1} worker processes that concurrently count the frequency of each value in the input array. Each worker process is started with a unique value to count, and by iterating over the input array, increments the corres from the channelponding position in the \texttt{parallel\_counts} array.

After all worker processes have completed, the main parallel counter process iterates throgh the results and saves the value with the highest frequency inside the \texttt{parallel\_result} variable.

To be able to detect when all worker processes have completed, I decided to use a \textit{channel} to synchronize the main process with the workers. The channel created in the main process and passed as an argument to each worker process. As soon as a worker process completes, it sends its PID through passed channel. As we start \texttt{MAX + 1} worker processes, we expect to read \texttt{MAX + 1} values from the channel. By leveraging this, the \textit{parallelCounter} process is able to detect when all worker processes have completed. The code is available in listing \ref{lst:promela_parallel_wait}.

\begin{lstlisting}[language=Promela, caption={ProMeLa parallel frequency counter - worker synchronization with channel}, captionpos=b, breaklines=true, label={lst:promela_parallel_wait}]
// 2) Parallel version
proctype parallelCounter() {
  // Create channel to wait for workers to finish
  chan joinCh = [MAX + 1] of { pid };
  // Create array of PID's for workers
  pid workers[MAX + 1];
  // Create a worker for each possible value
  int i;
  for (i : 0 .. MAX) {
    workers[i] = run parallelWorker(i, joinCh);
  }
  // Wait for all workers to finish by reading from the channel
  for (i : 0 .. MAX) {
    int done;
    joinCh ? done;
  }
  // If we are here, all workers are done as we read MAX+1 values from the channel
  printf("\t [!] all workers are done\n");
  ...
}
\end{lstlisting}

This implementation is different from the Java implementation, where threads are first created all together, then started, and joined in order to wait for their completion. In ProMeLa as we just saw, we start the worker processes right away, and we use a channel to wait for their completion.

\subsubsection{Worker Process}

\subsection{Parameters: MAX and LENGTH}

If MAX and LENGTH are too high, the model checking process can take a long time to complete, or even run out of memory. To avoid this, I decided to set the values of MAX and LENGTH to 2. This way, the model checking process is fast and efficient, and it is possible to verify the correctness of the program in a reasonable amount of time.

% Add screenshot of the model checking process with MAX / LENGTH too high
\pagebreak

\section{LTL Properties}
\label{sec:ltl_properties}

\subsection{Verify completition of both sequential and parallel processes}

To be able to verify the completition of both the sequential and parallel processes, I decided to use two global variables: \texttt{sequentialDone} and \texttt{parallelDone}. These variables are set to 0 by default, and set to 1 when the respective process has completed. To verify the completition of both processes, I defined the following LTL properties:

\begin{lstlisting}[language=Promela, caption={LTL properties to verify the completition of both sequential and parallel processes}, captionpos=b, breaklines=true, label={lst:ltl_properties}]
  ltl sequential_done { [] (sequentialDone == 1) }
  ltl parallel_done { [] (parallelDone == 1) }
\end{lstlisting}

\subsection{Additional LTL properties}

As requested by the assignment, I defined two additional LTL properties to verify the correctness of the program, one that must hold and one that must not hold. The LTL property that must hold is the following:

\begin{enumerate}
	\item \textbf{Property 1:} The sum of the frequencies of all values in the \texttt{sequential\_counts} array must be equal to the length of the input array.
	\item \textbf{Property 2:} The sum of the frequencies of all values in the \texttt{parallel\_counts} array must be equal to the length of the input array.
\end{enumerate}

% FIXME: Add a non-hold property!

\pagebreak

\section{Verification with SPIN}

% \bibliographystyle{abbrv}
% \bibliography{references}  % need to put bibtex references in references.bib 
\end{document}
